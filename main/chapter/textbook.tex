\chapter{物理教材推荐}

\section*{介绍}

为了给同学们一个更好的学习物理的指导,我想通过这一章,来给同学们在教材选择和参考方面提供一些建议,以达到更好的提升自我能力的目的。毫无疑问,我们需要接触各种书籍,而在繁杂的书籍中寻找到最合适的,是十分困难的,所以如此阅读指南就是想帮助同学们来更好地寻找到优秀的教材,以尽快的提升自己的能力。

以下,我们将推荐教材分为三个模块,一个是物理教材部分,一个是数学教材部分,另一个是其他科目的书籍(不限于教材、论文、小说),我们对每一个需要学习的模块所涉及的书目做一个汇总和基本的评价及其中的可取之处,并择其一二本书推荐为主修书目。

值得注意的是,物理和数学之间的关系是如此的密切,这就会导致一些书目难以界定其是物理教材还是数学教材(比如阿诺尔德的经典力学的数学方法,属于是理论力学中的异端),而这就会在分类上出现问题。

\section{数学部分}

\subsection*{分析}

\begin{enumerate}
  \item 陈天权:数学分析讲义

  非常现代,也比较难的一套数学分析教材,基本涵盖了数学分析,实分析,甚至流形上的微积分等内容,大而全。
\end{enumerate}

\subsection*{代数}

\begin{enumerate}
  \item Gilbert Strang: \textit{Introduction to Linear Algebra},

  图像十分清晰,较早引入线性空间的概念,讲解细致,适合入门,后半部分有较多应用,涉及内容较广,有作者本人的网课,对自学较为友好
  \item Michael Artin: \textit{Algebra}
  
  涵盖面较广,群环域等代数基本概念均有包含,轻证明,比较适合物理人阅读。
  
\end{enumerate}

\subsection*{数学物理方法}

\begin{enumerate}
  \item 梁昆淼:数学物理方法
  \item 顾樵:数学物理方法
  \item Hassani: \textit{Mathematical Physics}
  
  \item Arfen: \textit{Mathematical Methods for Physicists}
\end{enumerate}

\subsection*{其他书目}

\section{物理部分}

\subsection*{力学}

\begin{enumerate}
  \item 舒幼生:力学

  经典力学教材
\end{enumerate}

\subsection*{电磁学}

\begin{enumerate}
  \item 梁灿彬:电磁学(尤其是拓展篇)
\end{enumerate}

\subsection*{热学}

\begin{enumerate}
  \item Daniel V.Schroeder: \textit{Thermal Physics}
\end{enumerate}

\subsection*{光学}

\subsection*{原子物理学}

\subsection*{理论力学}

\begin{enumerate}
  \item Herbert Goldstein: \textit{Classical Mechanics}, 2nd

  格德斯坦的经典力学是一本十分经典的理论力学,其书的内容十分完善,知识体系也十分深入,当然这也对于读者的数学基础提出了一定的考验,值得作为主修教材。
  \item 朗道:《力学》

  朗道的力学是一本独特且经典的理论力学教材,其编写思路与大部分的理论力学有所区别,其教材以最小作用量原理为切点,逐步推导理论力学的整个体系,另外该教材内容及其凝练,叙述也不拖泥带水,通过阅读这本教材,可以体会到作者对于理论力学的独有的认识。但是,正因为其凝练的语言和独有的编写思路,所以它不适合作为初学教材,可以作为学习完理论力学之后的一个参考。
  \item 赵亚溥:《力学讲义》

  国科大的力学教材,采取经典力学、分析力学糅合的讲法,物理思想方面讲的相当精彩,尤其在思考题上可以深刻感受历史上物理学家的思想。旁征博引,从哈密顿-雅可比方程推导出薛定谔方程,提纲挈领,后半部分有有不少力学前沿知识。看这本书前若有一些普物的基础,那么看此书将如小说一样,趣味横生。
  \item 梅凤翔:《高等分析力学》

  完全"少儿不宜"的书,一本真正的力学的专著。基本能找到所有分析力学问题的答案,适合用来参考,非常不适合当作主修教材。
  \item 梁昆淼:《力学》 下册 理论力学

  梁老师的书十分经典,涵盖的内容十分广泛,包括基本的矢量力学、分析力学等以及连续介质力学等内容,同时也注重了基本理论的推导,同时经常结合实际模型问题提出问题,值得读者深入思考,非常适合初学者入门,可以作为一本主修书目。

  \item 陈童:《经典力学新讲》

  将能量与哈密顿量作为第一概念,篇幅较短

  \item 阿诺尔德:《经典力学的数学方法》

  严格来讲,这并不算是严格的理论力学书籍,其实是披着物理情景的数学书,因为其内容包含的相当深,比如群论、流形等更高级的数学知识,如果想要学习的话,必须要有相当深厚的代数学基础,把它看作是一本理论力学书籍实为是一种学术异端(真有人这么干的)

  \item S.T.Thornton ,J.B Marion: \textit{Classical Dynamics of Particles and Systems}

  Very extensive text with every essential concepts and exercises you should know, and never overloaded.
  You should be no problems in final exams if you nail examples in this book.
\end{enumerate}

\subsection*{电动力学\ 经典场论 }
\begin{enumerate}
  \item David J. Griffiths: \textit{Introduction to Electrodynamics}, 3rd

  该书实为电动力学入门级教材,对于毫无数理基础的同学(甚至是没有电磁学基础的同学)也能轻易理解该书的内容;本书会将所有需要的数学基础以十分简明的方式讲解出来,并加以Example去加深理解,而之后的电动力学的知识讲解十分详细,推导过程也易于理解,实为是电动教材的入门首选;唯一的不足就是对于电动力学的讲解深度有限,对于想要了解更多场论的知识的同学,显然有些不足,需要再配合其他层次更高的电动力学和场论的书籍来加深学习层次

  \item John David Jackson: \textit{Classical Electrodynamics}, 3rd.
  \item 周磊:电动力学讲义

  该书为讲述电动力学最好的中文教材之一。全书小而精,知识脉络很细致。推荐配合b站周老师的讲课视频一起使用。
\end{enumerate}


\subsection*{热力学与统计物理}

\begin{enumerate}
  \item Daniel V. Schroeder: An Introduction to Thermal Physics

  Excellent beginner to dive into the world of statistics physics. In fact, you can learn this book without any preliminaries of thermal physics. Intuitive and physical, although sometimes nonrigorous.
\end{enumerate}


\subsection*{量子力学(一、二)}

\begin{enumerate}
  \item David J. Griffiths: \textit{Introduction to Quantum Mechanics}

  A fair introduction. Textbook for babies.
  \item R. Shankar: Principles of Quantum Mechanics

  Good beginner on formalism of quantum mechanics. An uninteresting but clear book.

  \item S. Weinberg: Lectures on Quantum Mechanics

  Very good book! Hard to read, but you can always trust Weinberg, who never disappoint you. Both good in mathematical rigorous and physical picture. Some lovely note cannot be found anywhere else.
\end{enumerate}

\subsection*{固体物理}

\begin{enumerate}
  \item 黄昆:固体物理学
\end{enumerate}

\subsection*{狭义 \&广义相对论}

\begin{enumerate}
  \item 梁灿彬:微分几何入门与广义相对论
  
  国内最为经典的广义相对论教材了,其有非常全面的微分几何基础的讲解(前五章),在学习数学基础的时候,可以参考梁老的前五章,当然,梁书的数学语言风格有别于主流广相教材,即使用抽象符号表示,看个人喜好学习;在学习该书的时候,最好伴随着其网课一起食用,教材更像是讲义,精华还是在课程中;不过,梁书的一大问题是物理部分不够清晰,因此需要辅以其他教材。
  \item 赵峥:广义相对论基础
  
  这也是一本比较经典的广义相对论教材,不过教材本身的内容稍微简略,需要辅以课程视频
  \item 梁灿彬:从零学相对论
  
  一本对新手非常友好的相对论教材,基本上做到了从零开始的目的,讲解非常翔实且易懂,当然内容主要以狭义相对论为主,广义相对论部分的内容还是比较简略,更多的是科普性质的内容,需要更多的基础才可以学习广义相对论。
  \item Sean Carroll: \textit{Spacetime and Geometry An Introduction to General Relativity}
  
  卡罗尔的广相也是一本相当经典且“Normal” 的广相教材,内容充实;教材不会出现老旧的问题,以及使用的数学语言更适合初学,也不会像梁书一样需要铺垫过于厚重的数学基础,当然,教材部分内容的推导会有一些不够充分的问题,但作为初学者教材是完全满足需求的。
  \item MTW: Gravity
  
  国外广相教材的著名“大部头”,以内容讲解翔实充分,图例清晰简洁为特征,堪称广相教材中的标杆,可能唯一的缺点就是书太厚了。
\end{enumerate}


\subsection*{量子场论}
\subsubsection*{粒子物理方向}
\begin{enumerate}
  \item Peskin: An Introduction to Quantum Field Theory:

  To be written.
\end{enumerate}

\subsubsection*{凝聚态物理方向}
\begin{enumerate}
  \item Michael Stone: The Physics of Quantum Fields:
  A short but comprehensive introduction to QFT. It is a good book for beginners to learn QFT in condensed matter physics.
\end{enumerate}

\subsection*{其他书目}




